% À l'oral : dire pourquoi on peut vouloir utiliser des algorithmes gloutons

\begin{frame}{Points non traités}
    Je fais le choix de ne pas parler des approximations du Problème du Voyageur de Commerce parce qu'elles ont été traitées au TD Pratique 3.
\end{frame}

\begin{frame}{Algorithmes gloutons}
    \begin{block}{Algorithme glouton}
        Algorithme qui fait des choix localement optimaux pour construire une solution.
    \end{block}

    \begin{exampleblock}{Exemple : Kruskal}
        L'algorithme de Kruskal est un algorithme glouton : à chaque étape, il considère l'arête de \textbf{poids minimal} parmi celles restantes et ne créant pas de cycle.
        \medskip

        En l'occurrence, Kruskal est un algorithme optimal mais ce n'est pas toujours le cas !
    \end{exampleblock}
    \medskip

    Dans le cas général, on va donc chercher à borner le coût de la solution renvoyée par un algorithme pour garantir sa qualité.

\end{frame}

\begin{frame}{Ce qui est nouveau dans ce chapitre}
    \begin{block}{\newtext{$\alpha$-approximation}}
        Soit un problème d'optimisation, dont $f$ est la fonction de coût.
        \medskip

        On dit qu'un algorithme est une $\alpha$-approximation ($\alpha \geq 1$) pour ce problème si, pour toute entrée $e$, en notant $S$ la solution renvoyée par l'algorithme pour $e$ et $S^*$ une solution optimale :
        \[
            max\left(\frac{f(S)}{f(S^*)},\frac{f(S^*)}{f(S)}\right) \leq \alpha
        \]
    \end{block}
    \medskip

    \textbf{Minimisation} : $f(S)\leq \alpha f(S^*)$
    \smallskip

    \textbf{Maximisation} : $f(S) \geq \frac{1}{\alpha}f(S^*)$
    \medskip

    En pratique, on a souvent des solutions plus proches de l'optimale que de la borne.
\end{frame}

% Points importants pour l'examen
\begin{frame}{Conclusion de la section 4}
    Choses à savoir faire pour l'examen :
    \begin{itemize}
        \item Écrire un algorithme glouton
        \item Montrer qu'un algorithme est une $\alpha$-approximation d'un problème
    \end{itemize}
    \medskip

    Comment s'entraîner :
    \begin{itemize}
        \item Refaire le TD 4 (exercice 1)
    \end{itemize}
\end{frame}
