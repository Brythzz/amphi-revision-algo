\begin{frame}{Backtracking}

\end{frame}

% Slide sur le bruteforce (exemple avec une complexité de fou)
% Slide démo du backtracking
% Slide dans quels cas est-ce qu'on veut utiliser le backtracking
% Slide sur ce qu'apporte le branch and bound avec un mini exemple



\begin{frame}{Ce qui est nouveau dans ce chapitre}
    \begin{block}{\newtext{La programmation linéaire en nombres entiers}}
        Optimisation (maximisation/minimisation) d'une grandeur à coefficients entiers.
    \end{block}

    \begin{exampleblock}{Exemple de problème}
        Maximiser $Z=100x_1 + 150x_2$ sous les contraintes :
        \begin{itemize}
            \item $8000x_1 + 4000x_2 \leq 40000$
            \item $15x_1 + 30x_2 \leq 200$
            \item $x_1, x_2 \in \mathbb{N}$
        \end{itemize}
    \end{exampleblock}

    \textbf{Ce qu'il faut retenir} : on peut utiliser le branch and bound pour trouver la solution optimale dans ce genre de problèmes.
    \medskip

    La dernière question de l'exercice 2 du TD 3 traite de ce sujet.
\end{frame}

% Points importants pour l'examen
\begin{frame}{Conclusion de la section 3}
    Choses à savoir faire pour l'examen :
    \begin{itemize}
        \item Expliquer dans quels cas de figure on utilise le backtracking/branch and bound
        \item Écrire un algorithme de backtracking
        \item Donner la complexité des algorithmes
        % Je remets un point à part sur la complexité parce que c'est un peu plus complexe dans le cas des algos récursifs comme ici
    \end{itemize}
    \medskip

    Comment s'entraîner :
    \begin{itemize}
        \item Refaire le TD 3 (exercice 2, en entier)
    \end{itemize}
\end{frame}
