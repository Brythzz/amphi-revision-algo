\documentclass{beamer}

\usepackage[T1]{fontenc}
\usepackage[french]{babel}
\usepackage{lmodern}

\usepackage{club-scientifique}

% Supprimer des warnings
\hbadness=10000
\hfuzz=\maxdimen
\vfuzz=\maxdimen

% Librairies pour les figures Tikz
\usetikzlibrary{shapes.geometric,positioning,calc}

% Configuration des listings de code et pseudo-code

\usepackage{listingsutf8, algpseudocode}

\algrenewcommand\algorithmicwhile{\textbf{tant que}}
\algrenewcommand\algorithmicdo{\textbf{faire}}
\algrenewcommand\algorithmicend{\textbf{fin}}
\algrenewcommand\algorithmicif{\textbf{si}}
\algrenewcommand\algorithmicthen{\textbf{alors}}
\algrenewcommand\algorithmicelse{\textbf{sinon}}
\algrenewcommand\algorithmicfor{\textbf{pour}}
\algrenewcommand\algorithmicforall{\textbf{pour tout}}
\algrenewcommand\algorithmicrepeat{\textbf{répéter}}
\algrenewcommand\algorithmicuntil{\textbf{jusqu'à que}}
\algrenewcommand\algorithmicreturn{\textbf{retourner}}

\algnotext{EndFor}
\algnotext{EndIf}
\algnotext{EndWhile}
\algnotext{EndProcedure}

\renewcommand{\algorithmiccomment}[1]{%
  \hfill \textcolor{gray}{\small $\triangleright$ #1}%
}

% Styles Tikz pour les graphes
\tikzstyle{node_base}=[circle,thick,inner sep=0pt,minimum size=6mm,draw=blue!60,fill=blue!25]
\tikzstyle{node_selected}=[circle,thick,inner sep=0pt,minimum size=6mm,draw=red!50,fill=red!20]
\tikzstyle{node_visited}=[circle,thick,inner sep=0pt,minimum size=6mm,node distance=20mm,draw=black!50,fill=black!20]
\tikzstyle{directed_edge}=[->,>=stealth,semithick]
\tikzstyle{edge_selected}=[-,style=nearly transparent,line width=6pt]
\tikzstyle{small_node}=[circle,inner sep=1pt,minimum size=4mm,draw=blue!60,fill=blue!25]

% Coloration des cellules dans la démo de Dijkstra
\usepackage[table]{xcolor}

% Style de texte pour les nouveautés
\newcommand{\staricon}{\includegraphics[height=1.7ex]{images/sparkles}\space}
\newcommand{\newtext}[1]{\staricon #1 \staricon}


\title[Algorithmique et complexité]{Amphi de révision :\\ Algorithmique et complexité}
\author{Mattéo Rizza Murgier}
\titlegraphic{\includegraphics[width=0.2\textwidth]{images/logoClubScientifique.png}}
\date{19 janvier 2026}

\begin{document}
\frame{\titlepage}

\begin{frame}{Infos sur l'examen}
    \begin{itemize}
        \item L'examen dure 3h ;
        \item Vous avez le droit à des documents \textbf{manuscrits} ;
        \item Les chapitres 1 à 6, ainsi que les chapitres de pré-requis sont au programme de l'examen.
    \end{itemize}
    \bigskip

    \begin{alertblock}{Attention}
        La structure et le contenu du cours ont changé par rapport à l'année dernière. Certains concepts ont été ajoutés au cours et seront mis en évidence dans ce qui suit.
    \end{alertblock}
\end{frame}


\section{Algorithmes de graphes}
% % Graphe non-orienté
\begin{frame}{Définitions}
    \begin{block}{Définition - Graphe non-orienté}
        On note $G=(V,E)$ où :
        \begin{itemize}
            \item $V$ est un ensemble de n\oe{}uds ;
            \item $E$ est un ensemble d'arêtes ;
            \item Une arête $e=(u,v)$ est une paire d'éléments de $V$.
        \end{itemize}
    \end{block}
    \medskip
    \vspace{2.15em}

    \begin{center}
        \begin{tikzpicture}[scale=0.7]
            \node[small_node] (1) at (0,0) {\footnotesize{1}};
            \node[small_node] (4) at (-.7,-2) {\footnotesize{4}};
            \node[small_node] (3) at (1,-2) {\footnotesize{3}};
            \node[small_node] (8) at (2,-1) {\footnotesize{8}};
            \node[small_node] (2) at (3.5,0) {\footnotesize{2}};
            \node[small_node] (7) at (3.5,-1.5) {\footnotesize{7}};
            \node[small_node] (9) at (2.3,-3) {\footnotesize{9}};

            \draw[semithick] (1) to (2);
            \draw[semithick] (1) to (4);
            \draw[semithick] (1) to (3);
            \draw[semithick] (3) to (4);
            \draw[semithick] (3) to (9);
            \draw[semithick] (3) to (8);
            \draw[semithick] (8) to (7);
            \draw[semithick] (2) to node[auto]{\phantom{\footnotesize 2}} (7);
            \draw[semithick] (9) to (7);
        \end{tikzpicture}
    \end{center}

\end{frame}

% Graphe orienté
\begin{frame}{Définitions}
    \begin{block}{Définition - Graphe orienté}
        On note $G=(V,E)$ où :
        \begin{itemize}
            \item $V$ est un ensemble de n\oe{}uds ;
            \item $E$ est un ensemble d'arcs ;
            \item Une arête $e=(u,v)$ est un couple d'éléments de $V$.
        \end{itemize}
    \end{block}
    \medskip
    \vspace{2.15em}

    \begin{center}
        \begin{tikzpicture}[scale=0.7]
            \node[small_node] (1) at (0,0) {\footnotesize{1}};
            \node[small_node] (4) at (-.7,-2) {\footnotesize{4}};
            \node[small_node] (3) at (1,-2) {\footnotesize{3}};
            \node[small_node] (8) at (2,-1) {\footnotesize{8}};
            \node[small_node] (2) at (3.5,0) {\footnotesize{2}};
            \node[small_node] (7) at (3.5,-1.5) {\footnotesize{7}};
            \node[small_node] (9) at (2.3,-3) {\footnotesize{9}};

            \draw[directed_edge] (1) to (2);
            \draw[directed_edge] (1) to (4);
            \draw[directed_edge] (1) to (3);
            \draw[directed_edge] (3) to (4);
            \draw[directed_edge] (3) to (9);
            \draw[directed_edge] (3) to (8);
            \draw[directed_edge] (8) to (7);
            \draw[directed_edge] (2) to node[auto]{\phantom{\footnotesize 2}} (7);
            \draw[directed_edge] (9) to (7);
        \end{tikzpicture}
    \end{center}

\end{frame}

% Graphe pondéré
\begin{frame}{Définitions}
    \begin{block}{Définition - Graphe pondéré}
        On note $G=(V,E,w)$ où :
        \begin{itemize}
            \item $V$ est un ensemble de n\oe{}uds ;
            \item $E$ est un ensemble d'arêtes ;
            \item Une arête $e=(u,v)$ est une paire d'éléments de $V$.
            \item $w: E \rightarrow \mathbb{R}$ est une fonction de pondération
        \end{itemize}
    \end{block}
    \medskip

    \begin{center}
        \begin{tikzpicture}[scale=0.7]
            \node[small_node] (1) at (0,0) {\footnotesize{1}};
            \node[small_node] (4) at (-.7,-2) {\footnotesize{4}};
            \node[small_node] (3) at (1,-2) {\footnotesize{3}};
            \node[small_node] (8) at (2,-1) {\footnotesize{8}};
            \node[small_node] (2) at (3.5,0) {\footnotesize{2}};
            \node[small_node] (7) at (3.5,-1.5) {\footnotesize{7}};
            \node[small_node] (9) at (2.3,-3) {\footnotesize{9}};

            \draw[semithick] (1) to node[auto]{\footnotesize 3} (2);
            \draw[semithick] (1) to node[auto]{\footnotesize 1} (4);
            \draw[semithick] (1) to node[auto]{\footnotesize 2} (3);
            \draw[semithick] (3) to node[auto]{\footnotesize 4} (4);
            \draw[semithick] (3) to node[auto]{\footnotesize 1} (9);
            \draw[semithick] (3) to node[auto]{\footnotesize 3} (8);
            \draw[semithick] (8) to node[auto]{\footnotesize 1} (7);
            \draw[semithick] (2) to node[auto]{\footnotesize 2} (7);
            \draw[semithick] (9) to node[auto]{\footnotesize 3} (7);
	    \end{tikzpicture}
	\end{center}
\end{frame}

% Graphe pondéré orienté
\begin{frame}{Définitions}
    \begin{block}{Définition - Graphe orienté pondéré}
        On note $G=(V,E,w)$ où :
        \begin{itemize}
            \item $V$ est un ensemble de n\oe{}uds ;
            \item $E$ est un ensemble d'arêtes ;
            \item Une arête $e=(u,v)$ est une paire d'éléments de $V$.
            \item $w: E \rightarrow \mathbb{R}$ est une fonction de pondération
        \end{itemize}
    \end{block}
    \medskip

    \begin{center}
        \begin{tikzpicture}[scale=0.7]
            \node[small_node] (1) at (0,0) {\footnotesize{1}};
            \node[small_node] (4) at (-.7,-2) {\footnotesize{4}};
            \node[small_node] (3) at (1,-2) {\footnotesize{3}};
            \node[small_node] (8) at (2,-1) {\footnotesize{8}};
            \node[small_node] (2) at (3.5,0) {\footnotesize{2}};
            \node[small_node] (7) at (3.5,-1.5) {\footnotesize{7}};
            \node[small_node] (9) at (2.3,-3) {\footnotesize{9}};

            \draw[directed_edge] (1) to node[auto]{\footnotesize 3} (2);
            \draw[directed_edge] (1) to node[auto]{\footnotesize 1} (4);
            \draw[directed_edge] (1) to node[auto]{\footnotesize 2} (3);
            \draw[directed_edge] (3) to node[auto]{\footnotesize 4} (4);
            \draw[directed_edge] (3) to node[auto]{\footnotesize 1} (9);
            \draw[directed_edge] (3) to node[auto]{\footnotesize 3} (8);
            \draw[directed_edge] (8) to node[auto]{\footnotesize 1} (7);
            \draw[directed_edge] (2) to node[auto]{\footnotesize 2} (7);
            \draw[directed_edge] (9) to node[auto]{\footnotesize 3} (7);
        \end{tikzpicture}
	\end{center}
\end{frame}

% Graphe biparti
\begin{frame}{Graphes particuliers}
    \begin{block}{Graphe biparti}
        Graphe dont l'ensemble des sommets $V$ peut être partitionné en deux ensembles $V_1$ et $V_2$ tels que chaque arête a une extrémité dans $V_1$ et l'autre dans $V_2$.
    \end{block}
    \medskip

    \begin{center}
        \begin{tikzpicture}
            \node[small_node] (1) at (0,3) {\footnotesize{$v_{1,1}$}};
            \node[small_node] (2) at (0,2) {\footnotesize{$v_{1,2}$}};
            \node[small_node] (3) at (0,1) {\footnotesize{$v_{1,3}$}};
            \node[small_node] (4) at (0,0) {\footnotesize{$v_{1,4}$}};

            \node[small_node] (5) at (2,2.5) {\footnotesize{$v_{2,1}$}};
            \node[small_node] (6) at (2,1.5) {\footnotesize{$v_{2,2}$}};
            \node[small_node] (7) at (2,.5) {\footnotesize{$v_{2,3}$}};

            \draw[directed_edge] (1) to (5);
            \draw[directed_edge] (1) to (6);
            \draw[directed_edge] (2) to (6);
            \draw[directed_edge] (4) to (5);
            \draw[directed_edge] (4) to (7);
        \end{tikzpicture}
    \end{center}
\end{frame}

% Arbre
\begin{frame}{Graphes particuliers}
    \begin{block}{Arbre}
        Graphe acyclique et connexe (il existe un chemin entre toute paire de sommets).
    \end{block}
    \medskip

    \begin{center}
        \begin{tikzpicture}
            \node[small_node] (1) at (0,2) {\footnotesize{1}};
            \node[small_node] (2) at (-1,1) {\footnotesize{2}};
            \node[small_node] (3) at (1,1) {\footnotesize{3}};
            \node[small_node] (4) at (-2,0) {\footnotesize{4}};
            \node[small_node] (5) at (0,0) {\footnotesize{5}};

            \draw[semithick] (1) to (2);
            \draw[semithick] (1) to (3);
            \draw[semithick] (2) to (4);
            \draw[semithick] (2) to (5);
        \end{tikzpicture}
    \end{center}
\end{frame}

% DFS (principe)
\begin{frame}{Parcours en profondeur}
    \begin{block}{Idée de l'algorithme}
        Lorsqu'on visite un n\oe{}ud, on le marque comme visité, puis on visite immédiatement le premier de ses voisins qui n'a pas encore été vu, et ainsi de suite.
        \smallskip

        Complexité : $O(|V| + |E|)$ avec une liste d'adjacence
    \end{block}
    On peut soit utiliser une structure de pile (dernier arrivé, premier sorti), soit écrire une fonction récursive.
\end{frame}

% DFS (code)
\begin{frame}[fragile]{Parcours en profondeur}
    Pour un graphe sous forme de listes d'adjacence :
    \bigskip

    \lstinputlisting[language=Python]{scripts/dfs.py}
\end{frame}

% DFS (exemple)
\begin{frame}{Parcours en profondeur}
    \begin{center}
        \begin{tikzpicture}[scale=1.5]
            \only<1>{\node[node_base] (1) at (1.5,1) {1};}
            \only<2->{\node[node_visited] (1) at (1.5,1) {1};}

            \node[node_base] (2) at (0,0) {2};

            \only<-2>{\node[node_base] (3) at (2.5,0) {3};}
            \only<3->{\node[node_visited] (3) at (2.5,0) {3};}

            \only<-3>{\node[node_base] (9) at (1.5,-1) {9};}
            \only<4->{\node[node_visited] (9) at (1.5,-1) {9};}

            \only<-4>{\node[node_base] (7) at (4,0) {7};}
            \only<5->{\node[node_visited] (7) at (4,0) {7};}

            \only<-5>{\node[node_base] (8) at (4.5,1) {8};}
            \only<6->{\node[node_visited] (8) at (4.5,1) {8};}

            \only<-6>{\node[node_base] (5) at (3.5,2) {5};}
            \only<7->{\node[node_visited] (5) at (3.5,2) {5};}

            \only<-7>{\node[node_base] (4) at (3.5,1) {4};}
            \only<8->{\node[node_visited] (4) at (3.5,1) {4};}

            \only<-8>{\node[node_base] (6) at (4.5,2) {6};}
            \only<9->{\node[node_visited] (6) at (4.5,2) {6};}

            \draw[directed_edge] (1) -- (3);
            \draw[directed_edge] (1) -- (5);
            \draw[directed_edge] (2) -- (1);
            \draw[directed_edge] (3) -- (7);
            \draw[directed_edge] (3) -- (9);
            \draw[directed_edge] (4) -- (3);
            \draw[directed_edge] (4) -- (6);
            \draw[directed_edge] (5) -- (4);
            \draw[directed_edge] (7) -- (8);
        \end{tikzpicture}
    \end{center}
\end{frame}

% BFS (principe)
\begin{frame}{Parcours en largeur}
    \begin{block}{Principe de l'algorithme}
        Lorsqu'on visite un n\oe{}ud, on le marque comme visité, puis on visite successivement l'ensemble de ses voisins avant de passer à la suite.
        \smallskip

        Complexité : $O(|V| + |E|)$ avec une liste d'adjacence
    \end{block}
    On écrit une fonction itérative en utilisant une structure de file (premier arrivé, premier sorti).
\end{frame}

% BFS (code)
\begin{frame}[fragile]{Parcours en largeur}
    Pour un graphe sous forme de listes d'adjacence :
    \bigskip

    \lstinputlisting[language=Python]{scripts/bfs.py}
\end{frame}

% BFS (exemple)
\begin{frame}{Parcours en largeur}
    \begin{center}
        \begin{tikzpicture}[scale=1.5]
            \only<1>{\node[node_base] (1) at (1.5,1) {1};}
            \only<2->{\node[node_visited] (1) at (1.5,1) {1};}

            \node[node_base] (2) at (0,0) {2};

            \only<-2>{\node[node_base] (3) at (2.5,0) {3};}
            \only<3->{\node[node_visited] (3) at (2.5,0) {3};}

            \only<-3>{\node[node_base] (5) at (3.5,2) {5};}
            \only<4->{\node[node_visited] (5) at (3.5,2) {5};}

            \only<-4>{\node[node_base] (9) at (1.5,-1) {9};}
            \only<5->{\node[node_visited] (9) at (1.5,-1) {9};}

            \only<-5>{\node[node_base] (7) at (4,0) {7};}
            \only<6->{\node[node_visited] (7) at (4,0) {7};}

            \only<-6>{\node[node_base] (4) at (3.5,1) {4};}
            \only<7->{\node[node_visited] (4) at (3.5,1) {4};}

            \only<-7>{\node[node_base] (6) at (4.5,2) {6};}
            \only<8->{\node[node_visited] (6) at (4.5,2) {6};}

            \only<-8>{\node[node_base] (8) at (4.5,1) {8};}
            \only<9->{\node[node_visited] (8) at (4.5,1) {8};}

            \draw[directed_edge] (1) -- (3);
            \draw[directed_edge] (1) -- (5);
            \draw[directed_edge] (2) -- (1);
            \draw[directed_edge] (3) -- (7);
            \draw[directed_edge] (3) -- (9);
            \draw[directed_edge] (4) -- (3);
            \draw[directed_edge] (4) -- (6);
            \draw[directed_edge] (5) -- (4);
            \draw[directed_edge] (7) -- (8);
        \end{tikzpicture}
    \end{center}
\end{frame}

% Plus courts chemins
\begin{frame}{Plus courts chemins}
    \begin{block}{Optimalité de BFS}
        Pour des graphes \textbf{non-pondérés}, le parcours en largeur nous donne le plus court chemin entre le n\oe{}ud de départ et tous les autres n\oe{}uds du graphe.
    \end{block}

    Mais comment trouver le plus court chemin entre deux n\oe{}uds dans un graphe pondéré ?
\end{frame}

% Dijkstra (utilité)
\begin{frame}{Algorithme de Dijkstra}
    \begin{block}{Optimalité de Dijkstra}
        Pour des graphes \textbf{pondérés} à poids \textbf{positifs}, l'algorithme de Dijkstra nous donne le plus court chemin entre le n\oe{}ud de départ et tous les autres n\oe{}uds du graphe.
    \end{block}

    \begin{alertblock}{Attention !}
        Si le graphe contient des arêtes de poids strictement négatif, l'algorithme de Dijkstra peut retourner un chemin non-optimal.
    \end{alertblock}
\end{frame}

% Dijkstra (principe)
\begin{frame}{Algorithme de Dijkstra}
    \begin{block}{Principe de l'algorithme}
        On fait un parcours en largeur en visitant les voisins dans un ordre qui dépend du poids des arêtes. En pratique, on remplace la file classique par une file de priorité (tas min ou tas de Fibonacci).
        \smallskip

        Complexité : $O(|E|log(|V|))$ avec un tas binaire
    \end{block}
\end{frame}

% Dijkstra (pseudo-code)
\begin{frame}{Algorithme de Dijkstra}
    \textbf{Entrées} : Un graphe $G=(E,V,w)$, un sommet de départ $s$
    \medskip

    \textbf{Algorithme} :
    \begin{algorithmic}
        \State $distance[u] \gets +\infty$ pour tout sommet $u \in V$
        \State $distance[s] \gets 0$
        \State $parent[u] \gets u$ pour tout $u \in V$
        \State $F \gets$ File de priorité vide \Comment{Frontière}
        \State \textbf{Enfiler}(F, $(0, s)$)

        \State
        \While{non EstVide($F$)}
            \State $u \gets$ Defiler($F$) \Comment{Sommet avec $distance[u]$ minimum}

            \For{chaque voisin $v$ de $u$}
                \If{$distance[u] + w(u, v) < distance[v]$}
                    \State $distance[v] \gets distance[u] + w(u, v)$
                    \State $parent[v] \gets u$
                    \State \textbf{Mise à jour} de $v$ dans $F$ avec la priorité $distance[v]$
                \EndIf
            \EndFor
        \EndWhile

        \State \Return $(distance, parent)$
    \end{algorithmic}
\end{frame}

% Dijkstra (exemple)
\begin{frame}{Algorithme de Dijkstra}
    \begin{columns}
    \begin{column}{6cm}
        \begin{tikzpicture}[scale=.7]
            \node[node_base] (D) at (0,0) {D};
            \node[node_base] (A) at (2,2) {A};
            \node[node_base] (B) at (5,2) {B};
            \node[node_base] (E) at (2,-2) {E};
            \node[node_base] (F) at (5,-2) {F};
            \node[node_base] (C) at (7,0) {C};
            \draw[directed_edge] (D) to node[auto]{\footnotesize 3} (A);
            \draw[directed_edge] (D) to node[below]{\footnotesize 1} (E);
            \draw[directed_edge] (E) to[bend right] node[auto]{\footnotesize 1} (A);
            \draw[directed_edge] (A) to[bend right] node[auto]{\footnotesize 5} (E);
            \draw[directed_edge] (A) to node[auto]{\footnotesize 3} (B);
            \draw[directed_edge] (E) to node[auto]{\footnotesize 2} (B);
            \draw[directed_edge] (B) to node[auto]{\footnotesize 1} (F);
            \draw[directed_edge] (E) to node[auto]{\footnotesize 4} (F);
            \draw[directed_edge] (B) to node[auto]{\footnotesize 3} (C);
            \draw[directed_edge] (F) to node[auto]{\footnotesize 1} (C);
            %exploration à partir de D
            \only<2-3>{		\node[node_selected] (D) at (0,0) {D};}
            \only<3-4>{		\node[node_visited] (A) at (2,2) {A};
                            \node[node_visited] (E) at (2,-2) {E};
            }
            \only<3-6>{		\draw[edge_selected] (D) to (A);}
            \only<3->{		\draw[edge_selected] (D) to (E);}
            % exploration à partir de E
            \only<5-6>{		\node[node_selected] (E) at (2,-2) {E};}
            \only<6-7>{		\node[node_visited] (A) at (2,2) {A};
                            \node[node_visited] (B) at (5,2) {B};
                            \node[node_visited] (F) at (5,-2) {F};
            }
            \only<6->{		\draw[edge_selected] (E) to[bend right] (A);
                            \draw[edge_selected] (E) to (B);
            }
            \only<6-12>{	\draw[edge_selected] (E) to (F);}
            % exploration à partir de A
            \only<8-9>{		\node[node_selected] (A) at (2,2) {A};}
            \only<9>{		\node[node_visited] (B) at (5,2) {B};
                            \node[node_visited] (E) at (2,-2) {E};
                            \draw[edge_selected] (A) to[bend right] (E);
                            \draw[edge_selected] (A) to (B);
            }
            % exploration à partir de B
            \only<11-12>{	\node[node_selected] (B) at (5,2) {B};}
            \only<12-13>{	\node[node_visited] (C) at (7,0) {C};
                            \node[node_visited] (F) at (5,-2) {F};
            }
            \only<12-15>{	\draw[edge_selected] (B) to (C);}
            \only<12->{	\draw[edge_selected] (B) to (F);}
            % exploration à partir de F
            \only<14-15>{	\node[node_selected] (F) at (5,-2) {F};}
            \only<15-16>{	\node[node_visited] (C) at (7,0) {C};}
            \only<15->{	\draw[edge_selected] (F) to (C);}
            % exploration à partir de C
            \only<17>{	\node[node_selected] (C) at (7,0) {C};}
        \end{tikzpicture}
    \end{column}
    \begin{column}{4cm}
        {\footnotesize
            \begin{tabular}{|c|c|c|}
            \hline N\oe{}ud & Distance & Parent\\
        % A change en 4 et 7, pointeur en 8
            \hline \only<4,7>{\cellcolor{gray!25}}\only<8>{\cellcolor{red!25}}A
                & \only<4,7>{\cellcolor{gray!25}}\only<1-3>{\phantom{$\infty$}}\only<4-6>{3}\only<7>{1+1}\only<8->{2}
                & \only<4,7>{\cellcolor{gray!25}}\only<1-3>{\phantom{$\bullet$}}\only<4-6>{D}\only<7->{E}\\
        % B change en 7, pointeur en 11
            \hline \only<7>{\cellcolor{gray!25}}\only<11>{\cellcolor{red!25}}B
                & \only<7>{\cellcolor{gray!25}}\only<1-6>{\phantom{$\infty$}}\only<7>{1+2}\only<8->{3}
                & \only<7>{\cellcolor{gray!25}}\only<1-6>{\phantom{$\bullet$}}\only<7->{E}\\
        % C change en 13 et 16, pointeur en 17
            \hline \only<13,16>{\cellcolor{gray!25}}\only<17>{\cellcolor{red!25}}C
                & \only<13,16>{\cellcolor{gray!25}}\only<1-12>{\phantom{$\infty$}}\only<13>{3+3}\only<14-15>{6}\only<16>{4+1}\only<17->{5}
                & \only<13,16>{\cellcolor{gray!25}}\only<1-12>{\phantom{$\bullet$}}\only<13-15>{B}\only<16->{F}\\
        % D pointeur en 2
            \hline \only<2>{\cellcolor{red!25}}D
                & $0$
                & $\bullet$\\
        % E change en 4, pointeur en 5
            \hline \only<4>{\cellcolor{gray!25}}\only<5>{\cellcolor{red!25}}E
                & \only<4>{\cellcolor{gray!25}}\only<1-3>{\phantom{$\infty$}}\only<4->{1}
                & \only<4>{\cellcolor{gray!25}}\only<1-3>{\phantom{$\bullet$}}\only<4->{D}\\
        % F change en 7 et 13, pointeur en 14
            \hline \only<7,13>{\cellcolor{gray!25}}\only<14>{\cellcolor{red!25}}F
                & \only<7,13>{\cellcolor{gray!25}}\only<1-6>{\phantom{$\infty$}}\only<7>{1+4}\only<8-12>{5}\only<13>{3+1}\only<14->{4}
                & \only<7,13>{\cellcolor{gray!25}}\only<1-6>{\phantom{$\bullet$}}\only<7-12>{E}\only<13->{B}\\
            \hline
            \end{tabular}

            \bigskip
            \begin{tabular}{p{1.8cm}p{2cm}}
            Frontière = &
                \only<1>{$\{D\}$}
                \only<2>{$\{\}$}
                \only<3-4>{$\{A,E\}$}
                \only<5>{$\{A\}$}
                \only<6-7>{$\{A,B,F\}$}
                \only<8-10>{$\{B,F\}$}
                \only<11>{$\{F\}$}
                \only<12-13>{$\{C,F\}$}
                \only<14-16>{$\{C\}$}
                \only<17->{$\{\}$}
            \\
            \hspace{1.06cm}u = &
                \only<2-4>{$D$}
                \only<5-7>{$E$}
                \only<8-10>{$A$}
                \only<11-13>{$B$}
                \only<14-16>{$F$}
                \only<17->{$C$}
            \end{tabular}
        }
    \end{column}
    \end{columns}
\end{frame}

% Arbre couvrant minimal
\begin{frame}{Arbre couvrant minimal}
    \begin{block}{Arbre couvrant minimal (MST)}
        Soit un graphe non-orienté, connexe $G=(V,E)$. Un arbre couvrant minimal de $G$ (\textit{Minimum Spanning Tree}) est un sous-graphe acyclique de $G$ contenant tous les sommets de $G$ et tel que la somme des poids de ses arêtes est minimal.
    \end{block}

    \begin{block}{Kruskal}
        L'algorithme de Kruskal est un algorithme glouton qui permet de trouver un arbre couvrant minimal dans un graphe connexe.
        \medskip

        Complexité : $O(|E|log|V|)$ avec une structure Unir-Trouver
    \end{block}

    L'algorithme de Prim permet également de trouver un MST mais il est un peu plus complexe.
\end{frame}

% Kruskal (pseudo-code)
\begin{frame}{Algorithme de Kruskal}
        \textbf{Entrées} : Un graphe connexe pondéré $G=(E,V,w)$\\
        \textbf{Sortie} : Un ensemble $T\subseteq E$ décrivant un MST
    \medskip

    \textbf{Algorithme} :
    \begin{algorithmic}
        \State $T \gets \emptyset$
        \State $P \gets \textbf{Init}(|S|)$ \Comment{Structure de données Unir-Trouver}
        \State Trier les arêtes de $E$ par poids croissant
        \smallskip

        \For{$(u,v)\in E$ dans l'ordre}
            \If{$\textbf{Trouver}(P, u) \neq \textbf{Trouver}(P, v)$}
                \State Ajouter $(u,v)$ à $T$
                \State \textbf{Unir}(P, u, v)
            \EndIf
        \EndFor
        \smallskip

        \State\Return $T$
    \end{algorithmic}
\end{frame}

% Kruskal (exemple)
\begin{frame}{Algorithme de Kruskal}
    \begin{center}
        \begin{tikzpicture}[scale=2]
            % Étape 1
            \only<1>{\node[node_base] (F) at (1,.75) {F};}
            \only<2->{\node[node_visited] (F) at (1,.75) {F};}
            \only<1>{\node[node_base] (G) at (2.5,0) {G};}
            \only<2->{\node[node_visited] (G) at (2.5,0) {G};}
            \only<2->{\draw[edge_selected] (F) to (G);}

            % Étape 2
            \only<-2>{\node[node_base] (A) at (-.5,2) {A};}
            \only<3->{\node[node_visited] (A) at (-.5,2) {A};}
            \only<-2>{\node[node_base] (E) at (-.5,0) {E};}
            \only<3->{\node[node_visited] (E) at (-.5,0) {E};}
            \only<3->{\draw[edge_selected] (A) to (E);}

            % Étape 3
            \only<-3>{\node[node_base] (B) at (1,2) {B};}
            \only<4->{\node[node_visited] (B) at (1,2) {B};}
            \only<4->{\draw[edge_selected] (A) to (B);}

            % Étape 4
            \only<5->{\draw[edge_selected] (A) to (F);}

            % Étape 5
            \only<-5>{\node[node_base] (C) at (2.5,1.5) {C};}
            \only<6->{\node[node_visited] (C) at (2.5,1.5) {C};}
            \only<6->{\draw[edge_selected] (B) to (C);}

            % Étape 6
            \only<-6>{\node[node_base] (D) at (3,2.5) {D};}
            \only<7->{\node[node_visited] (D) at (3,2.5) {D};}
            \only<7->{\draw[edge_selected] (C) to (D);}

            \draw[semithick] (A) to node[auto]{\footnotesize 3} (B);
            \draw[semithick] (A) to node[auto]{\footnotesize 2} (E);
            \draw[semithick] (A) to node[auto]{\footnotesize 3} (F);
            \draw[semithick] (C) to node[auto]{\footnotesize 7} (B);
            \draw[semithick] (C) to node[auto]{\footnotesize 9} (D);
            \draw[semithick] (C) to node[auto]{\footnotesize 8} (G);
            \draw[semithick] (E) to node[auto]{\footnotesize 3} (G);
            \draw[semithick] (F) to node[auto]{\footnotesize 5} (B);
            \draw[semithick] (F) to node[auto]{\footnotesize 4} (E);
            \draw[semithick] (F) to node[auto]{\footnotesize 1} (G);
        \end{tikzpicture}
    \end{center}
\end{frame}

% Points importants pour l'examen
\begin{frame}{Conclusion de la section 1}
    Choses à savoir faire pour l'examen :
    \begin{itemize}
        \item Établir la complexité d'un algorithme
        \item Raisonner sur les propriétés des graphes
        \item Écrire des algorithmes impliquant BFS/DFS
        \item Faire tourner Dijkstra à la main sur un exemple
        \item Faire tourner Kruskal à la main sur un exemple pour trouver un MST
    \end{itemize}
    \medskip

    Comment s'entraîner :
    \begin{itemize}
        \item Refaire les TDs de pré-requis
        \item Faire tourner les algorithmes Dijkstra/Kruskal sur des exemples
    \end{itemize}
\end{frame}


\section{Complexité : problèmes P et NP}
% % Types de problèmes (décision / optimisation)
\begin{frame}{Types de problèmes}
    \begin{block}{Problèmes de décision}
        Problèmes auxquels on peut répondre par oui ou non.
        \begin{itemize}
            \item \textbf{Problème du plus court chemin} : existe-t-il un chemin de coût $\leq k$ entre deux n\oe{}uds $s$ et $t$ dans un graphe pondéré ?
            \item \textbf{Problème du cycle hamiltonien} : existe-t-il un cycle passant par tous les n\oe{}uds d'un graphe exactement une fois ?
        \end{itemize}
    \end{block}

    \begin{block}{Problèmes d'optimisation}
        Problèmes dans lesquels on cherche la solution de coût minimal (ou maximal).
        \begin{itemize}
            \item \textbf{Problème du plus court chemin} : quel est le chemin le plus court entre les n\oe{}uds $s$ et $t$ dans un graphe pondéré ?
            \item \textbf{Problème du voyageur de commerce} : trouver un cycle hamiltonien de poids minimum.
        \end{itemize}
    \end{block}
\end{frame}

% Méthode
\begin{frame}{Formaliser un problème}
    Pour formaliser un problème, on donne ses entrées (instances), et la question à laquelle on veut répondre.
    \bigskip
    \begin{exampleblock}{Exemple}
        Formaliser le problème du plus court chemin entre deux n\oe{}uds $s$ et $t$ dans un graphe pondéré sous forme de problème de décision.
        \medskip

        \textbf{Entrées} :
        \begin{itemize}
            \item Un graphe $G=(V,E,w)$ pondéré
            \item Des n\oe{}uds $s,t\in V^2$
            \item Un nombre $k\in \mathbb{N}$
        \end{itemize}

        \textbf{Question} : existe-t-il un chemin $C\subseteq E$ de $s$ à $t$ tel que :
        \[
            \sum_{c\in C} w(c) \leq k
        \]
    \end{exampleblock}
\end{frame}

\begin{frame}{Types de problèmes}
    \begin{alertblock}{Attention !}
        Dans tout ce qui suit, on s'intéresse uniquement à des \textbf{problèmes de décision}.
    \end{alertblock}
\end{frame}

% Classes (définitions)
\begin{frame}{Classes P et NP}
    Dans le cas des \textbf{problèmes de décision} uniquement, on peut définir les classes de complexité \textbf{P} et \textbf{NP}.

    \begin{block}{Classe P}
        Un problème est dans la classe P s'il peut être résolu par un algorithme en complexité polynomiale.
    \end{block}

    \begin{block}{Classe NP}
        Un problème est dans la classe NP si ses solutions peuvent être vérifiées par un algorithme en temps polynomial.
        \medskip

        Pour montrer qu'un problème est dans NP, il suffit de donner un algorithme de vérification et de montrer qu'il est polynomial.
    \end{block}

    On a naturellement $P \subseteq NP$.
\end{frame}

% Réduction polynomiale
\begin{frame}{Réductions de problèmes}
    \begin{block}{Instance et instance positive}
        Soit un problème $\mathcal{P}$. On note $\mathcal{D}$ les instances (ou entrées) possibles pour $\mathcal{P}$ (e.g l'ensemble des graphes).
        \smallskip

        On a $\mathcal{D} = \mathcal{D}^+ \sqcup \mathcal{D}^-$ (union disjointe) où :
        \begin{itemize}
            \item $\mathcal{D}^+$ est l'ensemble des entrées pour lesquelles la réponse au problème est oui (instances positives) ;
            \item $\mathcal{D}^-$ est l'ensemble des entrées pour lesquelles la réponse au problème est non (instances négatives).
        \end{itemize}
    \end{block}

    \begin{block}{Réduction}
        On dit qu'un problème $\mathcal{P}_1$ se réduit à un autre problème $\mathcal{P}_2$ s'il existe une fonction $tr$ qui transforme une entrée de $\mathcal{P}_1$ en entrée de $\mathcal{P}_2$ en temps polynomial, telle que $e \in \mathcal{D}_1^+$ ssi $tr(e) \in \mathcal{D}_2^+$.
    \end{block}
\end{frame}

% Schéma réduction
\begin{frame}{Réduction de problèmes}
    \begin{center}
        \includegraphics[width=0.6\textwidth]{images/reduction}
    \end{center}
\end{frame}

% Définitions
\begin{frame}{Problèmes NP-difficiles et NP-complets}
    \begin{block}{Problème NP-difficile}
        Un problème $\mathcal{P}_1$ est dit NP-difficile si tout problème de la classe NP peut s'y réduire.
        \medskip

        En pratique pour montrer que $\mathcal{P}_1$ est NP-complet, on trouve une réduction d'un problème NP-complet $\mathcal{P}_c$ à $\mathcal{P}_1$.
    \end{block}

    \begin{center}
        \begin{tikzpicture}
            \draw (0,0) ellipse [x radius=2cm, y radius=1.2cm];
            \node[text=gray] at (1.2,-.5) {NP};

            \node[inner sep=1pt] (Pc) at (-.2, .1) {$\mathcal{P}_c$};
            \node[inner sep=1pt] (P1) at (-.2, -.8) {$\mathcal{P}_1$};

            \draw[->] (-1.3, -.2) -- (Pc);
            \draw[->] (-1.2, .7) -- (Pc);
            \draw[->] (1.3, .4) -- (Pc);
            \draw[->] (Pc) -- (P1);
        \end{tikzpicture}
    \end{center}

    \begin{block}{Problème NP-complet}
        Un problème est dit NP-complet s'il est \textbf{dans la classe NP} et qu'il est \textbf{NP-difficile}.
    \end{block}
\end{frame}

% Exemple de réduction
\begin{frame}{Exemple de réduction}
    On se propose de réduire le problème \textsc{Chemin Hamiltonien} (NP-difficile) vers \textsc{Plus Long Chemin} pour montrer que \textsc{Plus Long Chemin} est NP-difficile.

    \begin{exampleblock}{Chemin Hamiltonien}
        \textbf{Entrées} : un graphe non orienté connexe $G=(V,E)$
        \smallskip

        \textbf{Question} : existe-t-il un chemin simple (sans cycle) qui visite tous les sommets de $G$ ?
    \end{exampleblock}

    \begin{exampleblock}{Plus long chemin}
        \textbf{Entrées} : un graphe non orienté pondéré connexe $G=(V,E,w)$ et un entier $k\in\mathbb{N}$
        \smallskip

        \textbf{Question} : existe-t-il un chemin simple $C$ dans $G$ tel que la somme des poids des arêtes de $C$ soit au moins $k$ ?
    \end{exampleblock}
\end{frame}

% Rédaction de la réduction
\begin{frame}{Exemple de réduction}
    On propose $tr(\langle G=(V,E) \rangle) = \langle G'=(V,E,w), k\rangle$ où $w(e)=1$ $\forall e \in E$ et $k=|V| - 1$.
    \medskip

    Il faut montrer que $\mathcal{I}_{Ham} \in \mathcal{D}_{Ham}^+ \iff tr(\mathcal{I}_{Ham}) \in \mathcal{D}_{PLC}^+$
    \medskip

    \boxed{\Rightarrow} Soit $\mathcal{I}_{Ham} = \langle G=(V,E) \rangle$ une instance positive du problème du chemin hamiltonien. Vu $\mathcal{I}_{Ham}\in\mathcal{D}_{Ham}^+$, il existe un chemin hamiltonien dans $G$, qui par définition contient $|V|-1$ arêtes. Dans $\mathcal{I}_{PLC}=tr(\mathcal{I}_{Ham})$, le chemin est de poids $|V|-1 = k$.\\
    Donc $\mathcal{I}_{PLC} \in \mathcal{D}_{PLC}^+$.
    \medskip

    \boxed{\Leftarrow} Réciproquement, si $\mathcal{I}_{PLC}$ est une instance positive du problème plus long chemin, il existe un chemin simple de poids au moins $k=|V|-1$. Comme chaque arête est de poids 1, il doit visiter au moins $k+1=|V|$ sommets. Le chemin étant simple (pas de répétition de sommets), on a un chemin hamiltonien.\\
    Donc $\mathcal{I}_{Ham} \in \mathcal{D}_{Ham}^+$.
    \medskip

    On a réduit Ham (NP-difficile) à PLC donc PLC est NP-difficile.
\end{frame}

% Méthode pour montrer la NP-complétude
\begin{frame}{Méthode pour montrer la NP-complétude}
    Pour montrer qu'un problème est \textbf{NP-complet} :
    \bigskip

    \begin{enumerate}
        \item On montre que le problème est \textbf{dans NP} en exhibant un algorithme de vérification de complexité polynomiale ;
        \item On montre que le problème est \textbf{NP-difficile} en réduisant un autre problème NP-difficile à lui ;
        \item On conclut en disant que NP \& NP-difficile $\iff$ NP-complet
    \end{enumerate}
\end{frame}

% Problèmes classiques (cours + TD)
\begin{frame}{Problèmes NP-complets classiques}
    \begin{exampleblock}{SAT}
        \textbf{Entrée} : une formule logique sous FNC e.g $(x_1 \lor x2) \land (\lnot x_1)$
        \smallskip

        \textbf{Question} : la formule est-elle satisfiable ?
    \end{exampleblock}

    \begin{exampleblock}{Stable}
        \textbf{Entrées} : un graphe non-orienté connexe $G=(V,E)$ ; $k\in \mathbb{N}$
        \smallskip

        \textbf{Question} : existe-t-il un stable $S$ de $G$, càd un ensemble de sommets qui ne sont pas reliés entre eux par une arête, tel que $|S| \geq k$ ?
    \end{exampleblock}

    \begin{exampleblock}{Vertex-Cover}
        \textbf{Entrées} : un graphe non-orienté connexe $G=(V,E)$ ; $k\in \mathbb{N}$
        \smallskip

        \textbf{Question} : existe-t-il un ensemble $V' \subseteq V$ de taille $|V'| \leq k$ tel que toute arête $(u,v) \in E$ a au moins une de ses extrémités dans $V'$ ($u \in V'$ ou $v \in V'$) ?
    \end{exampleblock}
\end{frame}

% Problèmes classiques (cours + TD)
\begin{frame}{Problèmes NP-complets classiques}
    \begin{exampleblock}{Set-Cover}
        \textbf{Entrées} : un ensemble d'éléments $U$ ; une famille $S \subset \mathcal{P}(U)$ de sous ensembles de $U$; un nombre $k\in \mathbb{N}$
        \smallskip

        \textbf{Question} : existe-il une sous-famille $S' \subseteq S$ telle que :
        \begin{itemize}
            \item $S'$ est une couverture de $U$, càd $U = \cup_{S_i \in S'}S_i$
            \item $card(S') \leq k$
        \end{itemize}
    \end{exampleblock}

    \begin{exampleblock}{Clique}
        \textbf{Entrées} : un graphe $G=(V,E)$ ; $k\in \mathbb{N}$
        \smallskip

        \textbf{Question} : existe-il un ensemble $S \subseteq V$ tel que :
        \begin{itemize}
            \item $\forall u,v\in S^2$, $(u,v) \in E$ (le sous graphe induit est complet)
            \item $|S| \geq k$
        \end{itemize}
    \end{exampleblock}
\end{frame}

% Points importants pour l'examen
\begin{frame}{Conclusion de la section 2}
    Choses à savoir faire pour l'examen :
    \begin{itemize}
        \item Formaliser/modéliser un problème (donner entrées et question)
        \item Donner la nature d'un problème (décision ou optimisation)
        \item Connaître les problèmes classiques et leur classe
        \item Montrer qu'un problème est NP-complet (dans NP + NP-difficile)
    \end{itemize}
    \medskip

    Comment s'entraîner :
    \begin{itemize}
        \item Refaire le TD 1 (tous les exercices), le TD 2 (toutes les questions sauf la 1.3 et la 2.2), et le TD 3 (exercice 1, questions 1 à 4).
    \end{itemize}
\end{frame}


\section{Algorithmes de résolution exacte}
% \begin{frame}{Backtracking}

\end{frame}

% Slide sur le bruteforce (exemple avec une complexité de fou)
% Slide démo du backtracking
% Slide dans quels cas est-ce qu'on veut utiliser le backtracking
% Slide sur ce qu'apporte le branch and bound avec un mini exemple



\begin{frame}{Ce qui est nouveau dans ce chapitre}
    \begin{block}{\newtext{La programmation linéaire en nombres entiers}}
        Optimisation (maximisation/minimisation) d'une grandeur à coefficients entiers.
    \end{block}

    \begin{exampleblock}{Exemple de problème}
        Maximiser $Z=100x_1 + 150x_2$ sous les contraintes :
        \begin{itemize}
            \item $8000x_1 + 4000x_2 \leq 40000$
            \item $15x_1 + 30x_2 \leq 200$
            \item $x_1, x_2 \in \mathbb{N}$
        \end{itemize}
    \end{exampleblock}

    \textbf{Ce qu'il faut retenir} : on peut utiliser le branch and bound pour trouver la solution optimale dans ce genre de problèmes.
    \medskip

    La dernière question de l'exercice 2 du TD 3 traite de ce sujet.
\end{frame}

% Points importants pour l'examen
\begin{frame}{Conclusion de la section 3}
    Choses à savoir faire pour l'examen :
    \begin{itemize}
        \item Expliquer dans quels cas de figure on utilise le backtracking/branch and bound
        \item Écrire un algorithme de backtracking
        \item Donner la complexité des algorithmes
        % Je remets un point à part sur la complexité parce que c'est un peu plus complexe dans le cas des algos récursifs comme ici
    \end{itemize}
    \medskip

    Comment s'entraîner :
    \begin{itemize}
        \item Refaire le TD 3 (exercice 2, en entier)
    \end{itemize}
\end{frame}


\section{Algorithmes d'approximation}
% % Algorithmes gloutons (def, exemples)
%       -> peut être optimal (e.g Kruskal) mais l'est pas forcément

% Exemple d'algo d'approx non glouton
% Définition de alpha-approx (nouveauté !)


\section{Programmation dynamique}
% \begin{frame}{La programmation dynamique}
    \begin{block}{Principe}
        Stocker le résultat de calculs intermédiaires qui interviennent plusieurs fois dans le résultat pour éviter de les calculer plusieurs fois.
    \end{block}
    \medskip

    Dès qu'on a une formule de récurrence d'ordre 2 ou plus, on doit penser à la programmation dynamique.
\end{frame}

\begin{frame}{Exemple : la suite de Fibonacci}
    \begin{exampleblock}{Suite de Fibonacci}
        $F(0) = F(1) = 1$\\
        $F(n) = F(n-1) + F(n-2)$
    \end{exampleblock}

    \begin{center}
        \begin{tikzpicture}[
        scale=.8,
        every node/.style={draw, rounded corners, font=\small, inner sep=2pt},
        level1/.style={sibling distance=60mm},
        level2/.style={sibling distance=28mm},
        level3/.style={sibling distance=15mm},
        level4/.style={sibling distance=12mm},
        edge from parent/.style={-latex, draw}
        ]

        \node (f5) {$F(5)$}
        child[level1] { node (f4) {$F(4)$}
            child[level2] { node (f3a) {$F(3)$}
            child[level3] { node (f2a) {$F(2)$}
                child[level4] { node {$F(1)$} }
                child[level4] { node {$F(0)$} }
            }
            child[level3] { node {$F(1)$} }
            }
            child[level2] { node (f2b) {$F(2)$}
            child[level3] { node {$F(1)$} }
            child[level3] { node {$F(0)$} }
            }
        }
        child[level1] { node (f3b) {$F(3)$}
            child[level2] { node (f2c) {$F(2)$}
            child[level3] { node {$F(1)$} }
            child[level3] { node {$F(0)$} }
            }
            child[level2] { node {$F(1)$} }
        };

        % Highlight repeated subproblems
        \begin{scope}
        \node[draw, dashed, fit=(f3a)] {};
        \node[draw, dashed, fit=(f3b)] {};

        \node[draw, dashed, fit=(f2a)] {};
        \node[draw, dashed, fit=(f2b)] {};
        \node[draw, dashed, fit=(f2c)] {};
        \end{scope}

        \end{tikzpicture}
    \end{center}
\end{frame}

\lstset{language=python, keywordstyle=\ttfamily\bf\color{csred}, basicstyle=\ttfamily\footnotesize, literate={\#}{{\#}}1 {é}{{\'e}}1, showspaces=false}

\begin{frame}[fragile, allowframebreaks]{Suite de Fibonacci}
    \begin{itemize}
        \item Approche naïve (récursive)
\begin{lstlisting}
def fib(n):
  if n==1 or n==2:
    return 1
  return fib(n-1)+fib(n-2)
\end{lstlisting}
        $\Rightarrow$ complexité exponentielle $O(\phi^n)$ ($\phi$ le nombre d'or)
        \bigskip

        \item Programmation dynamique (récursive avec mémoïsation)
\begin{lstlisting}
table = {0:0, 1:1}
def fib(n):
  if not n in table:
    table[n] = fib(n-1) + fib(n-2)
  return table[n]
\end{lstlisting}
        $\Rightarrow$ complexité linéaire $O(n)$
        \framebreak

        \item Programmation dynamique (itérative avec tabulation)
\begin{lstlisting}
table = {0:0, 1:1}
def fib(n):
    for i in range(2, n+1):
        table[i] = table[i-1] + table[i-2]
    return table[n]
\end{lstlisting}
        $\Rightarrow$ complexité linéaire $O(n)$
    \end{itemize}
    \bigskip

    Dans les exemples qui suivent et dans tout le cours, on utilise la programmation dynamique itérative (\textit{bottom up}).
\end{frame}

% Définition informelle de l'algo
\begin{frame}{Bellman-Ford}
    Un exemple d'algorithme très utile utilisant la programmation dynamique :
    \medskip

    \begin{block}{Algorithme de Bellman-Ford}
        Pour des graphes \textbf{pondérés} (même à poids négatifs), l'algorithme de Bellman-Ford nous donne le poids du plus court chemin entre un n\oe{}ud de départ $s$ et tous les autres n\oe{}uds du graphe.
    \end{block}
    \medskip

    L'algorithme a une moins bonne complexité temporelle que Dijkstra mais est plus général.
\end{frame}

\begin{frame}{Bellman-Ford}
    \begin{block}{Principe de l'algorithme}
        On stocke $OPT(i,v)$ la longueur du plus court chemin entre le n\oe{}ud départ et $v$, avec au plus $i$ arêtes.
        \medskip

        À chaque étape, on regarde s'il est plus court de garder le chemin à $i-1$ arêtes, ou d'utiliser un autre chemin déjà existant vers un voisin de $v$ et d'y ajouter l'arête qui le relie à $v$. ($\star$)
        \medskip

        Complexité : $O(|V|\times|E|)$ avec une liste d'adjacence, et $O(|V|^3)$ avec une matrice d'adjacence
    \end{block}
    \medskip

    Le choix ($\star$) est traduit par la formule :
    \[
    \hspace*{-.35cm}OPT(i,v) = \min\left(OPT(i-1,v), \min_{(u,v)\in E}(OPT(i-1,u) + w((u,v)))\right)
    \]
\end{frame}

% On dira que dans le cours c'est inversé

\begin{frame}[fragile]{Bellman-Ford : algorithme}
    \begin{algorithmic}
        \Function{Bellman-Ford}{$G=(V,E,w), s\in V$}
            \State $d[0, v] \gets +\infty$ pour tout sommet $u \in V$
            \State $d[0, s] \gets 0$

            \State
            \For{i allant de $1$ à $|V|-1$}
                \For{$v \in V$}
                    \State $d[i,v] = d[i-1,v]$
                    \For{$(u,v)\in E$}
                        \If{$d[i,v]>d[i-1,u] + w(u,v)$}
                            \State $d[i,v] = d[i-1,u] + w(u,v)$
                        \EndIf
                    \EndFor
                \EndFor
            \EndFor
            \Return $d$
        \EndFunction
    \end{algorithmic}
    \bigskip

    \begin{exampleblock}{Justification de la première boucle}
        Un chemin optimal ne peut contenir que $|V|-1$ arêtes au plus, sinon on aurait une répétition de sommets dans le chemin.
    \end{exampleblock}
\end{frame}

% Démo Bellman-Ford
\begin{frame}{Bellman-Ford : exemple}
    \begin{center}
        \tikzstyle{noeud}=[draw, shape=circle, minimum size=0.15\unitlength]
        \begin{tikzpicture}[scale=1.5]
        \node[noeud] (s0) at (0,0) {\(s_0\)};
        \node[noeud] (s1) at (2,0) {\(s_1\)};
        \node[noeud] (s2) at (4,0) {\(s_2\)};
        \node[noeud] (s3) at (1,1) {\(s_3\)};
        \node[noeud] (s4) at (3,1) {\(s_4\)};
        \node[noeud, fill=red!20] (s5) at (2,2) {\(s_5\)};

        \path[<-, font=\footnotesize]
        (s0) edge             node[above]       {\(-4\)}    (s1)
        (s0) edge[bend left]  node[left]       {\(-3\)}    (s5)
        (s1) edge             node[left]           {\(-2\)}    (s4)
        (s1) edge             node[right]         {\(-1\)}    (s3)
        (s2) edge             node[above]       {\(8\)}   (s1)
        (s2) edge[bend right]  node[right]   {\(3\)}    (s5)
        (s3) edge             node[right]         {\(6\)}   (s0)
        (s3) edge             node[left]           {\(4\)}   (s5)
        (s4) edge             node[right]         {\(2\)}   (s5)
        (s4) edge             node[left]           {\(-3\)}    (s2)
        ;

        \only<3>{\draw[directed_edge, ultra thick, bend right] (s5) to (s0);}
        \only<3>{\draw[directed_edge, ultra thick] (s1) to (s0);}
        \only<4>{\draw[directed_edge, ultra thick] (s3) to (s1);}
        \only<4>{\draw[directed_edge, ultra thick] (s4) to (s1);}
        \only<5>{\draw[directed_edge, ultra thick, bend left] (s5) to (s2);}
        \only<5>{\draw[directed_edge, ultra thick] (s1) to (s2);}
        \end{tikzpicture}
    \end{center}

    \begin{center}
        {\footnotesize
            \only<2>{
                \begin{tabular}{|c|c|c|c|c|c|c|}
                    \hline   & $s_0$ & $s_1$ & $s_2$ & $s_3$ & $s_4$ & $s_5$ \\
                    \hline 0 & $\infty$ & $\infty$ & $\infty$ & $\infty$ & $\infty$ & $0$ \\
                    \hline
                \end{tabular}

                \phantom{$s_0 = min(\infty,min(\infty-4,0-3))$}
            }
            \only<3>{
                \begin{tabular}{|c|c|c|c|c|c|c|}
                    \hline   & $s_0$ & $s_1$ & $s_2$ & $s_3$ & $s_4$ & $s_5$ \\
                    \hline 0 & $\infty$ & $\infty$ & $\infty$ & $\infty$ & $\infty$ & $0$ \\
                    \hline 1 & \cellcolor{gray!25}$-3$ & & & & & \\
                    \hline
                \end{tabular}

                \medskip
                $s_0 = min(\infty,min(\infty-4,0-3))$
            }
            \only<4>{
                \begin{tabular}{|c|c|c|c|c|c|c|}
                    \hline   & $s_0$ & $s_1$ & $s_2$ & $s_3$ & $s_4$ & $s_5$ \\
                    \hline 0 & $\infty$ & $\infty$ & $\infty$ & $\infty$ & $\infty$ & $0$ \\
                    \hline 1 & $-3$ & \cellcolor{gray!25}$\infty$ & & & & \\
                    \hline
                \end{tabular}

                \medskip
                $s_1 = min(\infty,min(\infty-1,\infty-2))$
            }
            \only<5>{
                \begin{tabular}{|c|c|c|c|c|c|c|}
                    \hline   & $s_0$ & $s_1$ & $s_2$ & $s_3$ & $s_4$ & $s_5$ \\
                    \hline 0 & $\infty$ & $\infty$ & $\infty$ & $\infty$ & $\infty$ & $0$ \\
                    \hline 1 & $-3$ & $\infty$ & \cellcolor{gray!25}3 & & & \\
                    \hline
                \end{tabular}

                \medskip
                $s_2 = min(\infty,min(\infty+8,0+3))$
        }
            \only<6>{
                \begin{tabular}{|c|c|c|c|c|c|c|}
                    \hline   & $s_0$ & $s_1$ & $s_2$ & $s_3$ & $s_4$ & $s_5$ \\
                    \hline 0 & $\infty$ & $\infty$ & $\infty$ & $\infty$ & $\infty$ & $0$ \\
                    \hline 1 & $-3$ & $\infty$ & $3$ & $4$ & $2$ & $0$ \\
                    \hline
                \end{tabular}

                \medskip
                \emph{on finit la ligne sur le même principe}
            }
            \only<7>{
                \begin{tabular}{|c|c|c|c|c|c|c|}
                    \hline   & $s_0$ & $s_1$ & $s_2$ & $s_3$ & $s_4$ & $s_5$ \\
                    \hline 0 & $\infty$ & $\infty$ & $\infty$ & $\infty$ & $\infty$ & $0$ \\
                    \hline 1 & $-3$ & $\infty$ & $3$ & $4$ & $2$ & $0$ \\
                    \hline 2 & $-3$ & $0$ & $3$ & $3$ & $0$ & $0$ \\
                    \hline
                \end{tabular}

                \medskip
                \emph{puis on fait la ligne suivante}
            }
            \only<8>{
                \begin{tabular}{|c|c|c|c|c|c|c|}
                    \hline   & $s_0$ & $s_1$ & $s_2$ & $s_3$ & $s_4$ & $s_5$ \\
                    \hline 0 & $\infty$ & $\infty$ & $\infty$ & $\infty$ & $\infty$ & $0$ \\
                    \hline 1 & $-3$ & $\infty$ & $3$ & $4$ & $2$ & $0$ \\
                    \hline 2 & $-3$ & $0$ & $3$ & $3$ & $0$ & $0$ \\
                    \hline 3 & $-4$ & $-2$ & $3$ & $3$ & $0$ & $0$ \\
                    \hline
                \end{tabular}

                \medskip
                \emph{et ainsi de suite\ldots}
            }
            \only<9>{
                \begin{tabular}{|c|c|c|c|c|c|c|}
                    \hline   & $s_0$ & $s_1$ & $s_2$ & $s_3$ & $s_4$ & $s_5$ \\
                    \hline 0 & $\infty$ & $\infty$ & $\infty$ & $\infty$ & $\infty$ & $0$ \\
                    \hline 1 & $-3$ & $\infty$ & $3$ & $4$ & $2$ & $0$ \\
                    \hline 2 & $-3$ & $0$ & $3$ & $3$ & $0$ & $0$ \\
                    \hline 3 & $-4$ & $-2$ & $3$ & $3$ & $0$ & $0$ \\
                    \hline 4 & $-6$ & $-2$ & $3$ & $2$ & $0$ & $0$ \\
                    \hline 5 & $-6$ & $-2$ & $3$ & $0$ & $0$ & $0$ \\
                    \hline
                \end{tabular}

                \phantom{...}
            }
        }
    \end{center}
\end{frame}

\begin{frame}{Comment trouver une relation de récurrence ?}
    Comment trouver la relation de récurrence pour un problème de programmation dynamique ?

    \begin{itemize}
        \item On commence par déterminer la dimension du problème (souvent 1D ou 2D)
        \item Trouver comment passer d'un état au suivant (la formule traduit souvent un choix)
        \item Identifier les cas de base
    \end{itemize}

    \begin{exampleblock}{Subset Sum}
        \textbf{Question} : existe-t-il un sous-ensemble de $A=[a_1, a_2, ..., a_n]$ de somme $T$ ?
        \medskip

        $dp[i][s] =$ True si on peut faire la somme $s$ avec les $i$ premiers éléments
        \medskip

        Relation de récurrence : $dp[i][s] = dp[i-1][s] \lor dp[i-1][s-a_i]$
        \smallskip

        Cas de base : $dp[0][0] =$ True
    \end{exampleblock}
\end{frame}

% Points importants pour l'examen
\begin{frame}{Conclusion de la section 5}
    Choses à savoir faire pour l'examen :
    \begin{itemize}
        \item Trouver une relation de récurrence dans un problème et écrire l'algorithme itératif correspondant
        \item Modifier un algorithme récursif pour ajouter de la mémoïsation
        \item Faire tourner Bellman-Ford sur un exemple
    \end{itemize}
    \medskip

    Comment s'entraîner :
    \begin{itemize}
        \item Refaire le TD 5 (sauf la question 2.5)
    \end{itemize}
\end{frame}


\section{Graphes de flots}
% Exemple de graphe de flot
\begin{frame}{Exemple de graphe de flot}
    \begin{tikzpicture}
        \node[node_base] (s) at (0,0) {$s$};
        \node[node_base] (r1) at (2,2) {$r_1$};
        \node[node_base] (r2) at (2,-2) {$r_2$};
        \node[node_base] (r3) at (5,2) {$r_3$};
        \node[node_base] (r4) at (5,-2) {$r_4$};
        \node[node_base] (t) at (7,0) {$t$};

        \path[directed_edge] (s) edge node[above,sloped] {\textcolor{blue}{11}/16} (r1);
        \path[directed_edge] (s) edge node[below,sloped] {\textcolor{blue}{8}/13} (r2);
        \path[directed_edge] (r1) edge node[above,sloped] {\textcolor{blue}{12}/12} (r3);
        \path[directed_edge] (r2) edge node[below,sloped] {\textcolor{blue}{1}/4} (r1);
        \path[directed_edge] (r2) edge node[below,sloped] (capa) {\textcolor{blue}{11}/14} (r4);
        \path[directed_edge] (r3) edge node[below,sloped] {\textcolor{blue}{4}/9} (r2);
        \path[directed_edge] (r3) edge node[above,sloped] {\textcolor{blue}{15}/20} (t);
        \path[directed_edge] (r4) edge node[below,sloped] {\textcolor{blue}{7}/7} (r3);
        \path[directed_edge] (r4) edge node[below,sloped] {\textcolor{blue}{4}/4} (t);

        \node[below left=1.5cm and 0.4cm of s,align=center] (slabel) {noeud\\source};
        \node[below right=1.5cm and 0.4cm of t,align=center] (tlabel) {noeud\\terminal};
        \draw[->,thick,red] (slabel) -- (s);
        \draw[->,thick,red] (tlabel) -- (t);

        \node[below =of capa] (capalabel) {\textcolor{blue}{flot} / capacité};
        \node[ellipse,draw,red,thick,minimum width=12mm,minimum height=8mm,inner sep=0pt] (circlecapa) at (capa.center) {};
        \draw[->, thick, red] (capalabel) -- (circlecapa);
    \end{tikzpicture}
\end{frame}

\begin{frame}{Propriétés du flot}
    Un flot $f$ réalisable doit vérifier les propriétés suivantes :

    \begin{block}{Règle flot-capacité}
        Le flot d'une arête ne peut pas dépasser sa capacité.
        \[
        \forall\, u,v\in V^2\quad 0 \leq f(u,v)\leq c(u,v)
        \]
    \end{block}

    \begin{block}{Règle de conservation du flot}
        À l'exception de $s$ et $t$, le flot entrant dans un n\oe{}ud est égal au flot en sortant.
        \[
            \forall\, u\in V\setminus\{s,t\}\quad\sum_{v\in V}f(u,v) = \sum_{v\in V} f(v,u)
        \]
    \end{block}
\end{frame}

\begin{frame}{Propriétés du flot}
    \begin{block}{Flot total}
        Le flot sortant de $s$ est égal à celui entrant dans $t$.
        \[
            \sum_{u\in V}f(s,u) = \sum_{u\in V}f(u,t)
        \]
    \end{block}

    \begin{block}{Valeur du flot}
        La valeur du flot $f$, notée $\varphi$ est donnée par :
        \[
            \varphi = \sum_{u\in V}f(s,u) = \sum_{u\in V}f(u,t)
        \]
    \end{block}

    Le problème du flot maximal consiste à trouver un flot $f$ ayant la valeur du flot maximale $\varphi_{max}$
\end{frame}

\begin{frame}{Algorithme de Ford-Fulkerson}
    \begin{block}{Ford-Fulkerson}
        L'algorithme de Ford-Fulkerson est un algorithme glouton qui permet de calculer le flot maximal dans un graphe de flot.
        \medskip

        Complexité : $O((|V|+|E|)\times \varphi_{max})$
    \end{block}

    \textbf{Remarque 1} : l'algorithme de Ford-Fulkerson n'impose aucune contrainte sur le type de parcours à effectuer. En pratique, on fait un DFS.
    \medskip

    \textbf{Remarque 2} : une variante de l'algorithme qui s'appuie sur le BFS (l'algorithme d'Edmonds-Karp) permet d'obtenir une complexité en $O(|V|\times|E|^2)$, indépendante de $\varphi_{max}$.
\end{frame}

\begin{frame}{Algorithme de Ford-Fulkerson}
    \begin{center}
        \tikzstyle{edge} = [draw,thick,-]
        \tikzstyle{selected edge} = [draw,line width=3pt,-,dodgerblue!50]
        \begin{tikzpicture}[scale=0.65,every node/.style={scale=0.8},baseline]

        \node at (3.5,3.5) {\em Graphe de flot};

        \node[node_base] (s) at (0,0) {$s$};
        \node[node_base] (r1) at (2,2) {$r_1$};
        \node[node_base] (r2) at (2,-2) {$r_2$};
        \node[node_base] (r3) at (5,2) {$r_3$};
        \node[node_base] (r4) at (5,-2) {$r_4$};
        \node[node_base] (t) at (7,0) {$t$};

        \path<beamer:1-3>[edge,->] (s)  -- node[above,sloped] {\textcolor{blue}{0}/16} (r1);
        \path<beamer:1-3>[edge,->] (s)  -- node[below,sloped] {\textcolor{blue}{0}/13} (r2);
        \path<beamer:1-3>[edge,->] (r1) -- node[above,sloped] {\textcolor{blue}{0}/12} (r3);
        \path<beamer:1-3>[edge,->] (r2) -- node[below,sloped] {\textcolor{blue}{0}/4} (r1);
        \path<beamer:1-3>[edge,->] (r2) -- node[below,sloped] {\textcolor{blue}{0}/14} (r4);
        \path<beamer:1-3>[edge,->] (r3) -- node[below,sloped] {\textcolor{blue}{0}/9} (r2);
        \path<beamer:1-3>[edge,->] (r3) -- node[above,sloped] {\textcolor{blue}{0}/20} (t);
        \path<beamer:1-3>[edge,->] (r4) -- node[below,sloped] {\textcolor{blue}{0}/4} (t);
        \path<beamer:1-3>[edge,->] (r4) -- node[below,sloped] {\textcolor{blue}{0}/7} (r3);

        \path<beamer:4-6>[edge,->] (s)  -- node[above,sloped] {\textcolor{blue}{4}/16} (r1);
        \path<beamer:4-6>[edge,->] (s)  -- node[below,sloped] {\textcolor{blue}{0}/13} (r2);
        \path<beamer:4-6>[edge,->] (r1) -- node[above,sloped] {\textcolor{blue}{4}/12} (r3);
        \path<beamer:4-6>[edge,->] (r2) -- node[below,sloped] {\textcolor{blue}{0}/4} (r1);
        \path<beamer:4-6>[edge,->] (r2) -- node[below,sloped] {\textcolor{blue}{4}/14} (r4);
        \path<beamer:4-6>[edge,->] (r3) -- node[below,sloped] {\textcolor{blue}{4}/9} (r2);
        \path<beamer:4-6>[edge,->] (r3) -- node[above,sloped] {\textcolor{blue}{0}/20} (t);
        \path<beamer:4-6>[edge,->] (r4) -- node[below,sloped] {\textcolor{blue}{4}/4} (t);
        \path<beamer:4-6>[edge,->] (r4) -- node[below,sloped] {\textcolor{blue}{0}/7} (r3);

        \path<beamer:7-9>[edge,->] (s)  -- node[above,sloped] {\textcolor{blue}{12}/16} (r1);
        \path<beamer:7-9>[edge,->] (s)  -- node[below,sloped] {\textcolor{blue}{0}/13} (r2);
        \path<beamer:7-9>[edge,->] (r1) -- node[above,sloped] {\textcolor{blue}{12}/12} (r3);
        \path<beamer:7-9>[edge,->] (r2) -- node[below,sloped] {\textcolor{blue}{0}/4} (r1);
        \path<beamer:7-9>[edge,->] (r2) -- node[below,sloped] {\textcolor{blue}{4}/14} (r4);
        \path<beamer:7-9>[edge,->] (r3) -- node[below,sloped] {\textcolor{blue}{4}/9} (r2);
        \path<beamer:7-9>[edge,->] (r3) -- node[above,sloped] {\textcolor{blue}{8}/20} (t);
        \path<beamer:7-9>[edge,->] (r4) -- node[below,sloped] {\textcolor{blue}{4}/4} (t);
        \path<beamer:7-9>[edge,->] (r4) -- node[below,sloped] {\textcolor{blue}{0}/7} (r3);

        \path<beamer:10-12>[edge,->] (s)  -- node[above,sloped] {\textcolor{blue}{12}/16} (r1);
        \path<beamer:10-12>[edge,->] (s)  -- node[below,sloped] {\textcolor{blue}{4}/13} (r2);
        \path<beamer:10-12>[edge,->] (r1) -- node[above,sloped] {\textcolor{blue}{12}/12} (r3);
        \path<beamer:10-12>[edge,->] (r2) -- node[below,sloped] {\textcolor{blue}{0}/4} (r1);
        \path<beamer:10-12>[edge,->] (r2) -- node[below,sloped] {\textcolor{blue}{4}/14} (r4);
        \path<beamer:10-12>[edge,->] (r3) -- node[below,sloped] {\textcolor{blue}{0}/9} (r2);
        \path<beamer:10-12>[edge,->] (r3) -- node[above,sloped] {\textcolor{blue}{12}/20} (t);
        \path<beamer:10-12>[edge,->] (r4) -- node[below,sloped] {\textcolor{blue}{4}/4} (t);
        \path<beamer:10-12>[edge,->] (r4) -- node[below,sloped] {\textcolor{blue}{0}/7} (r3);

        \path<beamer:13-15>[edge,->] (s)  -- node[above,sloped] {\textcolor{blue}{12}/16} (r1);
        \path<beamer:13-15>[edge,->] (s)  -- node[below,sloped] {\textcolor{blue}{11}/13} (r2);
        \path<beamer:13-15>[edge,->] (r1) -- node[above,sloped] {\textcolor{blue}{12}/12} (r3);
        \path<beamer:13-15>[edge,->] (r2) -- node[below,sloped] {\textcolor{blue}{0}/4} (r1);
        \path<beamer:13-15>[edge,->] (r2) -- node[below,sloped] {\textcolor{blue}{11}/14} (r4);
        \path<beamer:13-15>[edge,->] (r3) -- node[below,sloped] {\textcolor{blue}{0}/9} (r2);
        \path<beamer:13-15>[edge,->] (r3) -- node[above,sloped] {\textcolor{blue}{19}/20} (t);
        \path<beamer:13-15>[edge,->] (r4) -- node[below,sloped] {\textcolor{blue}{4}/4} (t);
        \path<beamer:13-15>[edge,->] (r4) -- node[below,sloped] {\textcolor{blue}{7}/7} (r3);

        \end{tikzpicture}
        \hfill
        \begin{tikzpicture}[scale=0.65,every node/.style={scale=0.8},baseline]

        \node at (3.5,3.5) {\em Graphe résiduel};

        \node[node_base] (s) at (0,0) {$s$};
        \node[node_base] (r1) at (2,2) {$r_1$};
        \node[node_base] (r2) at (2,-2) {$r_2$};
        \node[node_base] (r3) at (5,2) {$r_3$};
        \node[node_base] (r4) at (5,-2) {$r_4$};
        \node[node_base] (t) at (7,0) {$t$};

        \path<beamer:2>[edge,->] (s) -- node[above,sloped] {\textcolor{forestgreen}{16}} (r1);
        \path<beamer:3-4>[selected edge,->] (s) -- node[above,sloped] {\textcolor{forestgreen}{16}} (r1);
        \path<beamer:2-3>[edge,->] (s) -- node[below,sloped] {\textcolor{forestgreen}{13}} (r2);
        \path<beamer:2>[edge,->] (r1) -- node[above,sloped] {\textcolor{forestgreen}{12}} (r3);
        \path<beamer:3-4>[selected edge,->] (r1) -- node[above,sloped] {\textcolor{forestgreen}{12}} (r3);
        \path<beamer:2-3>[edge,->] (r2) -- node[below,sloped] {\textcolor{forestgreen}{4}}    (r1);
        \path<beamer:2>[edge,->] (r2) -- node[below,sloped] {\textcolor{forestgreen}{14}} (r4);
        \path<beamer:3-4>[selected edge,->] (r2) -- node[below,sloped] {\textcolor{forestgreen}{14}} (r4);
        \path<beamer:2>[edge,->] (r3) -- node[above,sloped] {\textcolor{forestgreen}{9}} (r2);
        \path<beamer:3-4>[selected edge,->] (r3) -- node[above,sloped] {\textcolor{forestgreen}{9}} (r2);
        \path<beamer:2-3>[edge,->] (r3) -- node[above,sloped] {\textcolor{forestgreen}{20}} (t);
        \path<beamer:2-3>[edge,->] (r4) -- node[below,sloped] {\textcolor{forestgreen}{7}} (r3);
        \path<beamer:2>[edge,->] (r4) -- node[below,sloped] {\textcolor{forestgreen}{4}} (t);
        \path<beamer:3>[selected edge,->] (r4) -- node[below,sloped] {\textcolor{forestgreen}{4}} (t);
        \path<beamer:4>[selected edge, color=csred,->] (r4) -- node[below,sloped] {\textcolor{csred}{4}} (t);

        \path<beamer:5>[edge,->         ] (s) -- node[above,sloped] {\textcolor{forestgreen}{12}} (r1);
        \path<beamer:6-7>[selected edge,->] (s) -- node[above,sloped] {\textcolor{forestgreen}{12}} (r1);
        \path<beamer:5-6>[edge,->       ] (s) -- node[below,sloped] {\textcolor{forestgreen}{13}} (r2);
        \path<beamer:5>[edge,->         ] ($(r1.east)+(+0mm,+1mm)$) -- node[above,sloped] {\textcolor{forestgreen}{8}} ($(r3.west)+(+0mm,+1mm)$);
        \path<beamer:6>[selected edge,->] ($(r1.east)+(+0mm,+1mm)$) -- node[above,sloped] {\textcolor{forestgreen}{8}} ($(r3.west)+(+0mm,+1mm)$);
        \path<beamer:7>[selected edge, color=csred,->] ($(r1.east)+(+0mm,+1mm)$) -- node[above,sloped] {\textcolor{csred}{8}} ($(r3.west)+(+0mm,+1mm)$);
        \path<beamer:5-6>[edge,->       ] ($(r3.west)+(+0mm,-1mm)$) -- node[below,sloped] {\textcolor{crimson}{4}} ($(r1.east)+(+0mm,-1mm)$);
        \path<beamer:5-6>[edge,->       ] (r2) -- node[below,sloped] {\textcolor{forestgreen}{4}}    (r1);
        \path<beamer:5-6>[edge,->       ] ($(r2.east)+(+0mm,-1mm)$) -- node[below,sloped] {\textcolor{forestgreen}{10}} ($(r4.west)+(+0mm,-1mm)$);
        \path<beamer:5-6>[edge,->       ] ($(r4.west)+(+0mm,+1mm)$) -- node[above,sloped] {\textcolor{crimson}{4}} ($(r2.east)+(+0mm,+1mm)$);
        \path<beamer:5-6>[edge,->       ] ($(r2.north east)+(+1mm,-1mm)$) -- node[below,sloped] {\textcolor{crimson}{4}} ($(r3.south west)+(+1mm,-1mm)$);
        \path<beamer:5-6>[edge,->       ] ($(r3.south west)+(-1mm,+1mm)$) -- node[above,sloped] {\textcolor{forestgreen}{5}} ($(r2.north east)+(-1mm,+1mm)$);
        \path<beamer:5>[edge,->         ] (r3) -- node[above,sloped] {\textcolor{forestgreen}{20}} (t);
        \path<beamer:6-7>[selected edge,->] (r3) -- node[above,sloped] {\textcolor{forestgreen}{20}} (t);
        \path<beamer:5-6>[edge,->       ] (r4) -- node[below,sloped] {\textcolor{forestgreen}{7}} (r3);
        \path<beamer:5-6>[lightgray,edge,->] (r4) -- node[below,sloped] {\textcolor{palegreen}{0}} (t);


        \path<beamer:8-9>[edge,->] (s) -- node[above,sloped] {\textcolor{forestgreen}{4}} (r1);
        \path<beamer:8>[edge,->] (s) -- node[below,sloped] {\textcolor{forestgreen}{13}} (r2);
        \path<beamer:9-10>[selected edge,->] (s) -- node[below,sloped] {\textcolor{forestgreen}{13}} (r2);
        \path<beamer:8-9>[lightgray,edge,->] ($(r1.east)+(+0mm,+1mm)$) -- node[above,sloped] {\textcolor{palegreen}{0}} ($(r3.west)+(+0mm,+1mm)$);
        \path<beamer:8-9>[edge,->] ($(r3.west)+(+0mm,-1mm)$) -- node[below,sloped] {\textcolor{crimson}{12}} ($(r1.east)+(+0mm,-1mm)$);
        \path<beamer:8-9>[edge,->       ] (r2) -- node[below,sloped] {\textcolor{forestgreen}{4}}    (r1);
        \path<beamer:8-9>[edge,->] ($(r2.east)+(+0mm,-1mm)$) -- node[below,sloped] {\textcolor{forestgreen}{10}} ($(r4.west)+(+0mm,-1mm)$);
        \path<beamer:8-9>[edge,->] ($(r4.west)+(+0mm,+1mm)$) -- node[above,sloped] {\textcolor{crimson}{4}} ($(r2.east)+(+0mm,+1mm)$);
        \path<beamer:8>[edge,->] ($(r2.north east)+(+1mm,-1mm)$) -- node[below,sloped] {\textcolor{crimson}{4}} ($(r3.south west)+(+1mm,-1mm)$);
        \path<beamer:9>[selected edge,->] ($(r2.north east)+(+1mm,-1mm)$) -- node[below,sloped] {\textcolor{crimson}{4}} ($(r3.south west)+(+1mm,-1mm)$);
        \path<beamer:10>[selected edge,color=csred,->] ($(r2.north east)+(+1mm,-1mm)$) -- node[below,sloped] {\textcolor{csred}{4}} ($(r3.south west)+(+1mm,-1mm)$);
        \path<beamer:8-9>[edge,->] ($(r3.south west)+(-1mm,+1mm)$) -- node[above,sloped] {\textcolor{forestgreen}{5}} ($(r2.north east)+(-1mm,+1mm)$);
        \path<beamer:8>[edge,->] (r3) -- node[above,sloped] {\textcolor{forestgreen}{12}} (t);
        \path<beamer:9-10>[selected edge,->] (r3) -- node[above,sloped] {\textcolor{forestgreen}{12}} (t);
        \path<beamer:8-9>[edge,->       ] (r4) -- node[below,sloped] {\textcolor{forestgreen}{7}} (r3);
        \path<beamer:8-9>[lightgray,edge,->] (r4) -- node[below,sloped] {\textcolor{palegreen}{0}} (t);

        \path<beamer:11-12>[edge,->] (s) -- node[above,sloped] {\textcolor{forestgreen}{4}} (r1);
        \path<beamer:11>[edge,->] (s) -- node[below,sloped] {\textcolor{forestgreen}{9}} (r2);
        \path<beamer:12-13>[selected edge,->] (s) -- node[below,sloped] {\textcolor{forestgreen}{9}} (r2);
        \path<beamer:11-12>[lightgray,edge,->] ($(r1.east)+(+0mm,+1mm)$) -- node[above,sloped] {\textcolor{palegreen}{0}} ($(r3.west)+(+0mm,+1mm)$);
        \path<beamer:11-12>[edge,->] ($(r3.west)+(+0mm,-1mm)$) -- node[below,sloped] {\textcolor{crimson}{12}} ($(r1.east)+(+0mm,-1mm)$);
        \path<beamer:11-12>[edge,->] (r2) -- node[below,sloped] {\textcolor{forestgreen}{4}}    (r1);
        \path<beamer:11>[edge,->] ($(r2.east)+(+0mm,-1mm)$) -- node[below,sloped] {\textcolor{forestgreen}{10}} ($(r4.west)+(+0mm,-1mm)$);
        \path<beamer:12-13>[selected edge,->] ($(r2.east)+(+0mm,-1mm)$) -- node[below,sloped] {\textcolor{forestgreen}{10}} ($(r4.west)+(+0mm,-1mm)$);
        \path<beamer:11-12>[edge,->] ($(r4.west)+(+0mm,+1mm)$) -- node[above,sloped] {\textcolor{crimson}{4}} ($(r2.east)+(+0mm,+1mm)$);
        \path<beamer:11-12>[edge,->] (r3) -- node[above,sloped] {\textcolor{forestgreen}{9}} (r2);
        \path<beamer:11>[edge,->] (r3) -- node[above,sloped] {\textcolor{forestgreen}{8}} (t);
        \path<beamer:12-13>[selected edge,->] (r3) -- node[above,sloped] {\textcolor{forestgreen}{8}} (t);
        \path<beamer:11>[edge,->       ] (r4) -- node[below,sloped] {\textcolor{forestgreen}{7}} (r3);
        \path<beamer:12>[selected edge,->       ] (r4) -- node[below,sloped] {\textcolor{forestgreen}{7}} (r3);
        \path<beamer:13>[selected edge,->,color=csred] (r4) -- node[below,sloped] {\textcolor{csred}{7}} (r3);
        \path<beamer:11-12>[lightgray,edge,->] (r4) -- node[below,sloped] {\textcolor{palegreen}{0}} (t);

        \path<beamer:14-15>[edge,->          ] (s) -- node[above,sloped] {\textcolor{forestgreen}{4}} (r1);
        \path<beamer:14-15>[edge,->          ] (s) -- node[below,sloped] {\textcolor{forestgreen}{2}} (r2);
        \path<beamer:14-15>[lightgray,edge,->] ($(r1.east)+(+0mm,+1mm)$) -- node[above,sloped] {\textcolor{palegreen}{0}} ($(r3.west)+(+0mm,+1mm)$);
        \path<beamer:14-15>[edge,->          ] ($(r3.west)+(+0mm,-1mm)$) -- node[below,sloped] {\textcolor{crimson}{12}} ($(r1.east)+(+0mm,-1mm)$);
        \path<beamer:14-15>[edge,->          ] (r2) -- node[below,sloped] {\textcolor{forestgreen}{4}}    (r1);
        \path<beamer:14-15>[edge,->          ] ($(r2.east)+(+0mm,-1mm)$) -- node[below,sloped] {\textcolor{forestgreen}{3}} ($(r4.west)+(+0mm,-1mm)$);
        \path<beamer:14-15>[edge,->          ] ($(r4.west)+(+0mm,+1mm)$) -- node[above,sloped] {\textcolor{crimson}{11}} ($(r2.east)+(+0mm,+1mm)$);
        \path<beamer:14-15>[edge,->          ] (r3) -- node[above,sloped] {\textcolor{forestgreen}{9}} (r2);
        \path<beamer:14-15>[edge,->          ] (r3) -- node[above,sloped] {\textcolor{forestgreen}{1}} (t);
        \path<beamer:14-15>[edge,->          ] ($(r3.south)+(-1mm,+0mm)$) -- node[below,sloped] {\textcolor{crimson}{7}} ($(r4.north)+(-1mm,+0mm)$);
        \path<beamer:14-15>[lightgray,edge,->] ($(r4.north)+(+1mm,+0mm)$) -- node[below,sloped] {\textcolor{palegreen}{0}} ($(r3.south)+(+1mm,+0mm)$);
        \path<beamer:14-15>[lightgray,edge,->] (r4) -- node[below,sloped] {\textcolor{palegreen}{0}} (t);
        
    \draw<beamer:15>[thick,dashed,csred] ($(r3.north west)+(-15mm,+4mm)$) -- ($(t.south west)+(-2mm,-14mm)$);

    \end{tikzpicture}
  \end{center}

  \medskip
  \onslide<15>{
    \begin{alertblock}{Terminaison}
        L'algorithme s'arrête lorsqu'il n'y a plus de chemin entre $s$ et $t$ dans le graphe résiduel
    \end{alertblock}
  }
\end{frame}

% À l'oral, on pourra dire que la coupe sur la slide au dessus est de capacité 23 = flot max
\begin{frame}{Théorème Max-flow Min-cut}
    \begin{block}{Coupe $s-t$}
        Partition des n\oe{}uds du graphe $V$ en $S$ et $T=V\setminus S$ telle que $s\in S$ et $t \in T$.
        \medskip

        Sa capacité est donnée par :
        \[
            c(S,T) = \sum_{(u,v)\in S\times T} c(u,v)
        \]
    \end{block}

    \begin{block}{Théorème Max-flow Min-cut}
        Les trois propositions sont équivalentes :
        \begin{enumerate}
            \item Le flot $\varphi$ entre $s$ et $t$ est maximal ;
            \item Il n'existe aucun chemin augmentant ;
            \item Il existe une coupe $s-t$ dont la capacité est égale à $\varphi$.
        \end{enumerate}
    \end{block}
\end{frame}

\begin{frame}{Modélisation de problèmes par un graphe de flot}
    Les problèmes d'affectation ou de répartition de ressources peuvent très souvent être modélisés par des graphes de flots avec une composante bipartie.

    \begin{exampleblock}{Répartition de l'eau de puits entre des villes}
    \begin{center}
        \begin{tikzpicture}[scale=.8] \footnotesize
        \node (f1) at (0, -1) {$A$};
        \node (f2) at (0, -3) {$B$};
        \node (f3) at (0, -5) {$C$};

        \node (p1) at (4, 0) {$D$};
        \node (p2) at (4, -2) {$E$};
        \node (p3) at (4, -4) {$F$};
        \node (p4) at (4, -6) {$G$};

        \node (s) at (-3, -3) {$s$};
        \node (t) at (7, -3) {$t$};

        \begin{scope}[every path/.style={->}]
        \draw (f1) -- (p1) node[midway, above,sloped] {10};
        \draw (f1) -- (p2) node[near end, above,sloped] {15};
        \draw (f1) -- (p3) node[near end, above,sloped] {10};
        \draw (f1) -- (p4) node[near end, above,sloped] {25};
        \draw (f2) -- (p1) node[near end, above,sloped] {20};
        \draw (f2) -- (p4) node[near end, above,sloped] {5};
        \draw (f2) -- (p3) node[near end, above,sloped] {15};
        \draw (f3) -- (p2) node[near start, above,sloped] {10};
        \draw (f3) -- (p3) node[near start, above,sloped] {5};
        \draw (f3) -- (p4) node[near start, above,sloped] {10};

        \draw (s) -- (f1) node[midway, above,sloped] {45};
        \draw (s) -- (f2) node[midway, above,sloped] {25};
        \draw (s) -- (f3) node[midway, above,sloped] {30};
        \draw (p1) -- (t) node[midway, above,sloped] {30};
        \draw (p2) -- (t) node[midway, above,sloped] {10};
        \draw (p3) -- (t) node[midway, above,sloped] {20};
        \draw (p4) -- (t) node[midway, above,sloped] {30};
        \end{scope}

        \end{tikzpicture}
    \end{center}
    \end{exampleblock}
\end{frame}

\begin{frame}{Modélisation de problèmes par un graphe de flot}
    \begin{exampleblock}{Répartition de ressources}
        Pour les problèmes de répartition de ressources, on peut dégager une structure générale :
        \begin{itemize}
            \item Une composante bipartie au centre dont les arêtes modélisent les contraintes de répartition (e.g le débit max d'un tuyau entre un puits et une ville)
            \item Des arêtes entre $s$ et les sources, modélisant les quantités de ressources allouables
            \item Des arêtes entre les destinations et $t$, modélisant les quantités cibles
        \end{itemize}
    \end{exampleblock}
\end{frame}

\begin{frame}{Méthode générale de résolution d'un problème}
    Pour résoudre un problème se modélisant par un graphe de flot (souvent de répartition/affectation de ressources), on suit les étapes suivantes :
    \begin{itemize}
        \item Modéliser le problème par un graphe de flot
        \item Faire tourner l'algorithme de Ford-Fulkerson pour trouver le flot maximal
        \item Extraire la solution du problème du graphe de flots rempli
    \end{itemize}
\end{frame}

% Points importants pour l'examen
\begin{frame}{Conclusion de la section 6}
    Choses à savoir faire pour l'examen :
    \begin{itemize}
        \item Modéliser un problème par un graphe de flot
        \item Faire tourner Ford-Fulkerson à la main sur un exemple pour trouver la solution d'un problème
        \item Connaître et pouvoir raisonner sur les propriétés du flot
    \end{itemize}
    \medskip

    Comment s'entraîner :
    \begin{itemize}
        \item Refaire le TD 6 (exercices 1 et 2)
    \end{itemize}
\end{frame}


\end{document}



%%%%%%%% Graphes
% Définitions de base
% Parcours en profondeur -> exécution + code
% Parcours en largeur -> exécution + code
% Dijkstra -> seulement exécution

%%%%%%%% Arbre couvrant minimal (MST = minimal spanning tree)
% Kruskal -> exécution + code

%%%%%%%% Méthodes de résolution exactes
% Backtracking
% Branch and bound

%%%%%%%% Algos gloutons
% Exemple d'algorithme glouton avec code
% Définition de alpha-approx + exemple

% 


% Ça peut être cool de mettre des annotations avec les exercices de TDs qui correspondent si les gens veulent les retravailler

%%%%%%%% Problèmes NP
% Formalisation de problèmes : quand on vous demande de donner une modélisation du pb, de formaliser le pb, etc.. On attend les entrées du pb et la question qu'on se pose.
% Les problèmes classiques de TD (Vertex Cover, Set Cover, etc...)

%%%%%%%% Programmation dynamique
% Exemples

%%%%%%%% Algos de flot
% 