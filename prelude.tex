% Supprimer des warnings
\hbadness=10000
\hfuzz=\maxdimen
\vfuzz=\maxdimen

% Librairies pour les figures Tikz
\usetikzlibrary{shapes.geometric,positioning,calc}

% Configuration des listings de code et pseudo-code

\usepackage{listingsutf8, algpseudocode}

\algrenewcommand\algorithmicwhile{\textbf{tant que}}
\algrenewcommand\algorithmicdo{\textbf{faire}}
\algrenewcommand\algorithmicend{\textbf{fin}}
\algrenewcommand\algorithmicif{\textbf{si}}
\algrenewcommand\algorithmicthen{\textbf{alors}}
\algrenewcommand\algorithmicelse{\textbf{sinon}}
\algrenewcommand\algorithmicfor{\textbf{pour}}
\algrenewcommand\algorithmicforall{\textbf{pour tout}}
\algrenewcommand\algorithmicrepeat{\textbf{répéter}}
\algrenewcommand\algorithmicuntil{\textbf{jusqu'à que}}
\algrenewcommand\algorithmicreturn{\textbf{retourner}}

\algnotext{EndFor}
\algnotext{EndIf}
\algnotext{EndWhile}
\algnotext{EndProcedure}

\renewcommand{\algorithmiccomment}[1]{%
  \hfill \textcolor{gray}{\small $\triangleright$ #1}%
}

% Styles Tikz pour les graphes
\tikzstyle{node_base}=[circle,thick,inner sep=0pt,minimum size=6mm,draw=blue!60,fill=blue!25]
\tikzstyle{node_selected}=[circle,thick,inner sep=0pt,minimum size=6mm,draw=red!50,fill=red!20]
\tikzstyle{node_visited}=[circle,thick,inner sep=0pt,minimum size=6mm,node distance=20mm,draw=black!50,fill=black!20]
\tikzstyle{directed_edge}=[->,>=stealth,semithick]
\tikzstyle{edge_selected}=[-,style=nearly transparent,line width=6pt]
\tikzstyle{small_node}=[circle,inner sep=1pt,minimum size=4mm,draw=blue!60,fill=blue!25]

% Coloration des cellules dans la démo de Dijkstra
\usepackage[table]{xcolor}

% Style de texte pour les nouveautés
\newcommand{\staricon}{\includegraphics[height=1.7ex]{images/sparkles}\space}
\newcommand{\newtext}[1]{\staricon #1 \staricon}
